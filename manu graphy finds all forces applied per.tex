manu graphy finds all forces applied perpendicuarly to cylindrical surface,

3 diff cylindrical surfaces on 3 diff days

divided into 7 anatomical areas

precent calc = contact area relation with load applied

load distribution of dominant and non dominant hand with 3 diff cylindrical sizes(circumference - 100,150 and 200mm) are observed

load distribution changed across all the fingers

manugraphy system (3 cylinders covered with pressure sensor matrices and evaluate forces applied perpedicular to cylindrical object)

in this paper they have pre-recorded audio instructions to grip and release

Manugraphy analyzer electronics is used to transfer the data to the computer

this software calc the perpendicular data for each sensor and summ = s1 + s2 +s3 .... the data

this is the raster diag of load applied at each area and latter is the force over time diagram

total 9 values are recorded for each hand with 3 cylinders at 3 intervels

percentage is calculated by assuming force applied on overall contact area is 100% and contribution of each of the 7 sections to total is calulcated.

individual load applied by cylinders:
 - total contact area is 100% and then

 - each finger contrib is calc across 150 and 200 mm cylinders 

 - the load applied by dominant and non dominat hand are compared 

 in the next scenario , each finger is subdivided it into 4 masks

 the contact area of each finger is set to 100% and then individual loads on each section is caluculated...

 the results state that 1/3rd of the load is applied by index and 1/3 by middle finger and the other part is contributed by both ring and small fingers

 load distribution b/w a person's dominant and non dominant hand are same

 strength in thumd and index finger increases inc with cylinder size and 